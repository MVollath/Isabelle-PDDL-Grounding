% !TeX root = ../main.tex

\chapter{Background}
\todo{General stuff about this project, maybe repeat that I do PDDL into STRIPS instead of into SASP.}
\section{Abstract Planning Language Syntax}

\subsection{PDDL}
The Planning Domain Definition Language (PDDL) is the de facto standard artificial intelligence planning language and is commonly used in planning competitions.
A PDDL task defines variables with which parameterized predicates and actions can be instantiated. PDDL is commonly referred to as a first-order planning language because e.g. action preconditions are first-order formulas over the predicates.
There are multiple versions and levels of PDDL and we are concerned with the abstract syntax specified in \cite{AbLa}, which my implementation builds upon.
It is very similar to the syntax used by Helmert but differs in a few ways which I will highlight.
PDDL tasks are commonly divided into domain and problem.

\subsubsection{Abstract Syntax}
A PDDL domain is a tuple $\langle\mathcal T, \mathcal P, \mathcal C, \mathcal A\rangle$.

The set of \textit{primitive} types and their hierarchy are defined by the directed graph $\mathcal T$. Every node corresponds to a type and the edge $(a, b)$ signifies that $a$ directly inherits $b$.
\cite{AbLa} supports Either types. Hence, a type is a list of primitive type alternatives. Intuitively, this means that certain variables can be instantiated by differently typed objects.

$\mathcal C$ is the set of constants; a set of symbols that are assigned a type. Although \cite{AbLa} allow constants to have multiple types, I restrict them to singular types instead, in order to be consistent with Helmert and Correa.

$\mathcal P$ is the set of predicate symbols. Each predicate symbol additionally has a corresponding signature expressed as a list of types. A predicate can be instantiated with an appropriate list of constants or variables (see below) to form a binary variable called atom. A ground atom is an atom that only refers to constants.

A state is a set of ground atoms and thus defines a valuation of these binary variables.

$\mathcal A$ is the set of actions. An action $A$ consists of a parameter list $\mathit{params}(A)$, a precondition $\mathit{pre}(A)$ and an effect, which itself consists of two sets of atoms $\mathit{adds}(A)$ and $\mathit{dels}(A)$.
The parameter list defines variables and assigns a type to each one of them.
The precondition is a formula over predicate atoms instantiated with domain constants and/or parameter variables.
Helmert allows these formulas to contain first-order quantifiers, but \cite{AbLa} restricts them to propositional formulas.
The atoms in the effect are likewise instantiated with constants and/or variables.
In other formalizations, the action parameters are often implicitly defined via the open variables in the precondition and the effect. However, I decided against that due to ambiguities that may arise from the use of Either types and for the sake of simplicity.

A PDDL problem is a tuple $\langle\mathcal D, \mathcal O, \mathcal I, \mathcal G\rangle$. $\mathcal D$ simply refers to a domain and $\mathcal O$ (O for objects) is an additional set of constants, like $\mathcal C$.

The initial state $\mathcal I$ is simply a state, and the goal $\mathcal G$ is a propositional formula over ground atoms.

Helmert does not distinguish between domain and problem, but that has no bearing on the semantics.
However, Helmert allows for the definition of axioms (\todo{explain}), even though Correa et al later omit them. Axioms are sets of rules that define new predicate atoms based on which atoms are true in a given state.

\subsubsection{Semantics}

Actions can refer to constants from $\mathcal C$ but not those from $\mathcal O$. Other than that, there is no semantic bearing on whether a constant is defined at the domain-level or the problem-level.




\subsection{STRIPS}


\subsection{Datalog}

\section{Grounding Process by Helmert}

\section{Grounding Process by Correa}